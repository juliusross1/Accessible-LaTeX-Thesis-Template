%Todo
%Table
%Colors




\documentclass[12pt,reqno]{amsbook}


%%\usepackage[english]{babel}  %Language needs to be specified for accessibility; currently this is not working well with latexml

% Math
\usepackage{amsmath, amssymb, amsthm}

% If you are including computer code the following package is useful.  It is similar to the verbatim package
\usepackage{listings} 



%Graphics
\usepackage{graphicx}

% Bibliography
%\usepackage[backend=biber,style=numeric,maxbibnames=99]{biblatex}
%\addbibresource{thesis.bib}

% Layout
\usepackage{geometry}
\geometry{margin=1in}

% Theorem environments
\newtheorem{theorem}{Theorem}[chapter]
\newtheorem{lemma}[theorem]{Lemma}
\theoremstyle{definition}
\newtheorem{definition}[theorem]{Definition}

% Hyperref
\usepackage[pdfusetitle]{hyperref}

\begin{document}

% Title metadata.  Make changes here to fit your needs
\newcommand{\thetitle}{A Sample Thesis in Mathematics}
\title{\thetitle}
\date{\today}
\newcommand{\institution}{University of Illinois Chicago} 
\newcommand{\degree}{Doctor of Philosophy}
\newcommand{\advisor}{Prof.~Ada Lovelace}
\newcommand{\theauthor}{A.~Student}
\author{\theauthor}

\frontmatter

% Custom title page
% Take care in making substantial changes to this as it needs to still work with latexml
\begin{titlepage}
    \centering
    {\Large \institution\par}
    \vspace{2cm}
    {\Large \thetitle \par}
    \vspace{2cm}
    {\Large by\par}
    \vspace{0.5cm}
    {\Large \theauthor\par}
    \vfill
    A thesis submitted in partial fulfillment of the requirements \\
    for the degree of  \degree \\
    \vspace{0.5cm}
    Advisor: \advisor \\
    \vspace{1cm}
    \makeatletter\@date \makeatother
\end{titlepage}

\begin{abstract}
This thesis investigates XYZ. We show results about ...
\end{abstract}

\tableofcontents

\mainmatter

\chapter{Introduction}
This is the introduction chapter. We cite some classic works \cite{Hartshorne,Mumford}.

\begin{theorem}\label{thm1}
This is a theorem
\end{theorem}

We reference Theorem \ref{thm1}.

\href{https://www.uic.edu}{University of Illinois Chicago}.

% JPG, JPEG, PNG will work.  PNG and SVG do not work.
\section{Motivation}


\begin{figure}\label{fig1}
\includegraphics[alt="Description of Image that serves the same purpose",scale=0.3]{torus.jpg}
\caption{This is a torus}
\end{figure}



\begin{figure}\label{fig2}
\includegraphics[alt="Description of Image that serves the same purpose",scale=0.8]{figure.png}
\caption{The Snake Lemma}
\end{figure}

%% PDF images do not work
%\begin{figure}\label{fig1}
%\includegraphics[alt="Description of Image that serves the same purpose",scale=0.3]{torus.pdf}
%\caption{This is a torus pdf}
%\end{figure}

\subsection{Historical context}
A brief overview of how the problem developed.
\begin{equation}\label{eq1}\int_0^1 f(x) dx = 2\end{equation}
How to solve \eqref{eq1}
\subsection{Open questions}
Some questions remain open for future work.

\section{Outline of the thesis}
We summarize the structure of the thesis.

\chapter{Background}
This chapter gives necessary background.

\section{Group theory}
\begin{definition}
A group is a set $G$ with a binary operation satisfying closure, associativity, identity, and inverses.
\end{definition}

\begin{theorem}
Every finite subgroup of the multiplicative group of a field is cyclic.
\end{theorem}

\begin{proof}
This is a standard result from algebra.
\end{proof}

\chapter{Main Results}
Here we present the main contributions of the thesis.

\section{A computer simulation}

% The following works with LaTeXml.  I consider the result accessible, but it might not be 100% WCAG compliant.
\begin{lstlisting}[basicstyle = \ttfamily\small,resetmargins=true,tabsize=5,extendedchars=false]
def factorial(n):
    """Compute the factorial of n recursively."""
    if n == 0:
        return 1
    else:
        return n * factorial(n - 1)

print(f"5! = {factorial(5)}")
\end{lstlisting}



\section{Second main result}
Another significant theorem.

\appendix
\chapter{Technical Lemmas}
Here we collect some supporting lemmas.

\begin{thebibliography}{9}


\bibitem{Hartshorne}
R.~Hartshorne, \emph{Algebraic Geometry}, Springer-Verlag, New York, 1977.

\bibitem{Mumford}
D.~Mumford, \emph{Abelian Varieties}, Oxford University Press, 1970.

\bibitem{DraismaEtAl}
J.~Draisma, E.~Horobet, G.~Ottaviani, B.~Sturmfels, and R.~R.~Thomas, 
``The Euclidean distance degree of an algebraic variety,'' 
\emph{arXiv:1309.0049} (2013).  
Available at: \href{https://arxiv.org/abs/1309.0049}{https://arxiv.org/abs/1309.0049}

\end{thebibliography}


\end{document}
